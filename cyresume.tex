%!TEX program = xelatex
\documentclass[11pt,a4paper,nolmodern]{moderncv}
\usepackage{xeCJK}
\usepackage{info}
\defaultfontfeatures{Ligatures=TeX}
\setmainfont{Minion Pro}
\setsansfont[SlantedFont=* Italic, BoldFont=* Bold]{Minion Pro}
%\setsansfont[SlantedFont=* Italic]{Myriad Pro}
%\setCJKmainfont[BoldFont = SimHei, ItalicFont = KaiTi]{SimSun}
%\setCJKsansfont{SimHei}
%\setCJKmonofont{FangSong}
\setCJKmainfont[BoldFont = SimHei, ItalicFont = KaiTi]{SimSun}
\setCJKsansfont{KaiTi}
\setCJKmonofont{KaiTi}

\title{西安电子科技大学计算机学院}
\myquote{励学,笃行,厚德,求真。}{}

\begin{document}


\hyphenpenalty=10000
\maketitle

\section{教育背景}
\tlcventry{2012}{0}{西安电子科技大学,2015年毕业}{}{}{}{
%\begin{itemize}
 % \item 学位:工学硕士 \qquad 学院:计算机学院 \qquad 专业:计算机系统结构 \@\qquad 导师:崔江涛
%\end{itemize}
\begin{itemize}
 \item \onech{学位:工学硕士}{学\quad\quad 院:计算机学院}{专业:计算机系统结构}{导\quad\quad 师:崔江涛}
\end{itemize}
}
\vspace{0.5em}
\tlcventry{2008}{2012}{西安电子科技大学}{}{}{}{
%\begin{itemize}
%  \item 学位:工学学士 \qquad 学院:计算机学院 \qquad 专业:计算机科学与技术 \qquad 专业排名:20\%
%\end{itemize}}
\begin{itemize}
 \item \onech{学位:工学学士}{学\quad\quad 院:计算机学院}{专业:计算机科学与技术}{专业排名:Top 20\%}
\end{itemize}}

\section{能力}
\subsection{开发}
\cvlistitem{\small 熟练使用C/C++,Matlab,掌握C\#,了解JAVA,Python,Lua,Go等语言}
\cvlistitem{\small 掌握基本算法,数据结构。熟悉Windows环境编程}
\cvlistitem{\small 掌握OpenCV库,能使用OpenCV编写基本的图像处理程序}
\cvlistitem{\small 掌握OpenGL库,能使用OpenGL进行基本的三维模型编程}
\cvlistitem{\small 掌握Hadoop平台,能编写基于MapReduce计算框架的程序,了解Pig,Hive,Hbase的一些应用}
\cvlistitem{\small 了解Linux环境与内核,并能在该环境进行基本的编程}

\subsection{其他}
% \cvcomputer{Office}{iWork, OpenOffice/LibreOffice, Microsoft Office}
%            {操作系统}{Mac OS X, GNU/Linux(Ubuntu, Mint), Windows}
% \cvcomputer{排版}{\XeLaTeX{}}
%            {编辑器}{Sublime Text,VIM}
\cvlistitem{
	\onech{\small 代码管理:Git,SVN}{\small 数据库:SQL Server,MongoDB}{\small 排\quad\quad 版:\XeLaTeX{}}{\small 编辑器:Sublime Text,VIM}
	% \twoch{代码管理:Git,SVN}{数据库:SQL Server,MongoDB}
	% \onech{代码管理:Git,SVN}{数据库:SQL Server,MongoDB}{}{}
	% \twoch{代码管理:Git,SVN}{数据库:SQL Server,MongoDB}
}
% \cvlistitem{
% 	% \onech{排\quad\quad 版:\XeLaTeX{}}{编辑器:Sublime Text,VIM}{}{}
% 	\twoch{排\quad\quad 版:\XeLaTeX{}}{编辑器:Sublime Text,VIM}
% }

\section{项目经历}
\tlcventry{2013}{0}{影像姿态判读软件}{}{}{}{}
\cvhobby{\textbf{项目描述}}{本项目为某基地无人机三维姿态测量的一套软件系统,主要利用光电经纬仪获得序列影像数据和航迹数据,以及结合光电经纬仪自身的姿态角(旋转角和俯仰角)以及空间坐标等数据信息,测量具体的飞机飞行姿态(俯仰角、翻滚角和偏航角)。}
\cvhobby{\textbf{责任描述}}{主要负责轮廓线提取算法的设计与实现、轮廓线匹配算法的设计与实现、系统测试。}
\vspace{0.5em}
\tldatecventry{2013.06}{校园云音乐平台}{}{}{}{}
\cvhobby{\textbf{项目描述}}{本平台结合当下流行的云计算技术,进行音乐的分布式存储和播放,保障音乐存储的拓展性和安全性,同时利用分布式计算框架实现音乐的热度排行榜,并结合当下的机器学习技术,挖掘用户可能喜欢的音乐并推荐。并能根据用户收听歌曲的历史数据,为用户推荐兴趣相类似的其他用户。参加华为产品孵化计划。}
\cvhobby{\textbf{责任描述}}{平台搭建,推荐算法设计。}
\vspace{0.5em}
\tldatecventry{2012.11}{基于LSH和LSB方法的高维数据索引研究}{}{}{}{}
\cvhobby{\textbf{项目描述}}{本项目旨在通过减少磁盘I/O次数、优化索引结构、数据降维的方法来提高高维数据索引的效率。}
\cvhobby{\textbf{责任描述}}{完成了LSB的数据结构设计以及LSB和LSH的程序实现以及效率分析。}
\vspace{0.5em}
\tldatecventry{2012.04}{基于小波分析与结构相关性的图像增强算法的设计与实现}{}{}{}{}
\cvhobby{\textbf{项目描述}}{通过小波分析的方法,对图像进行多层小波分解,分析分解后各层图像的结构相关性,来达到去除噪声,增强图像的目的。毕业设计。}
\cvhobby{\textbf{责任描述}}{个人独立完成,包括算法与框架的设计和实现。}
\vspace{0.5em}
\tldatecventry{2011.05}{魔兽世界商业插件}{}{}{}{}
\cvhobby{\textbf{项目描述}}{本项目基于Lua语言,制作与魔兽世界游戏里经济操作相关的插件,旨在方便游戏里的玩家,该插件属于游戏管理条例的合法范围。}
\cvhobby{\textbf{责任描述}}{编写拍卖行各物品价格分析、物品快捷拍卖等功能,各频道自动喊话功能。}
\vspace{0.5em}
\tldatecventry{2010.11}{宿舍管理软件}{}{}{}{}
\cvhobby{\textbf{项目描述}}{为方便宿舍同学日常生活,编写了该软件。参加校星火杯。}
\cvhobby{\textbf{责任描述}}{个人独立编写完成,包括新闻模块、邮件模块、订水模块、宿舍卫生检查模块等。}
\vspace{0.5em}
\tldatecventry{2010.09}{全国大学生数学建模大赛}{}{}{}{}
\cvhobby{\textbf{项目描述}}{为参加该比赛,前期做了基本的训练,9月10日参加比赛。}
\cvhobby{\textbf{责任描述}}{各数学模型的算法的设计与实现。}
\vspace{0.5em}
\tldatecventry{2010.11}{班级论坛}{}{}{}{}
\cvhobby{\textbf{项目描述}}{为活跃自己班级的气氛,增强班级的学术氛围,故在Discuz模板下做了这个班级论坛。}
\cvhobby{\textbf{责任描述}}{使用当时比较火的PHP+Mysql+Apache组合,主要负责数据库模块和每周主题讨论模块。}

\section{所获奖励}
\mycvlanguage{2012.09}{校奖学金}{一等奖}
\mycvlanguage{2010.10}{2010年全国大学生数学建模竞赛}{省级一等奖}
\mycvlanguage{2010.04}{校ACM竞赛}{三等奖}
\mycvlanguage{2010.12}{校星火杯}{三等奖}
\mycvlanguage{2009.09}{校奖学金/院优秀学生}{三等奖}

\section{语言}
\mycvlanguage{英语}{CET6}{428}
\mycvlanguage{英语}{CET4}{445}

\section{个人兴趣}
\cvhobby{运动}{足球,乒乓球,羽毛球,登山}
\cvhobby{互联网}{\href{http://yusion.sinaapp.com/}{个人博客},\href{http://blog.csdn.net/cyxcw1}{CSDN博客},
				 \href{http://weibo.com/u/2284680571}{新浪微博},\href{https://github.com/cyxcw1}{GitHub}}
\cvhobby{其他}{PC游戏,电影,读书}

\end{document}
